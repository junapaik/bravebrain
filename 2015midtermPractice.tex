% These lines can probably stay unchanged, although you can remove the last
% two packages if you're not making pictures with tikz.
\documentclass[11pt]{exam}
\RequirePackage{amssymb, amsfonts, amsmath, latexsym, verbatim, xspace, setspace}
\RequirePackage{tikz, pgflibraryplotmarks}

% By default LaTeX uses large margins.  This doesn't work well on exams; problems
% end up in the "middle" of the page, reducing the amount of space for students
% to work on them.
\usepackage[margin=1in]{geometry}
\usepackage{kotex}


% Here's where you edit the Class, Exam, Date, etc.
\newcommand{\class}{SUPERNOVA 고1반}
\newcommand{\term}{2015 봄학기 모의 중간고사}
\newcommand{\examdate}{4/10/2015}
\newcommand{\timelimit}{50 분}

% For an exam, single spacing is most appropriate
%\singlespacing
\onehalfspacing
% \doublespacing

% For an exam, we generally want to turn off paragraph indentation
\parindent 0ex

\begin{document} 
	% These commands set up the running header on the top of the exam pages
\pagestyle{head}
\firstpageheader{}{}{}
\runningheader{\class}{\examnum\ - Page \thepage\ of \numpages}{\examdate}
\runningheadrule
	
\begin{flushright}
	\begin{tabular}{p{2.8in} r l}
		\textbf{\class} & \textbf{Name (Print):} & \makebox[2in]{\hrulefill}\\
		\textbf{\term} &&\\
		\textbf{\examnum} &&\\
		\textbf{\examdate} &&\\
		\textbf{시험시간: \timelimit} 
	\end{tabular}\\
\end{flushright}
\rule[1ex]{\textwidth}{.1pt}

문제를 잘 읽고 답을 골라 답란에 표기하시오.\\
	
서술형 문항은 별도의 서술형 문항 답안지에, 검정색이나 파란색 필기구를 사용하여 작성하시오.\\
	
총문항수:  선택형 (16) 문항, 서술형 (4)문항\\
	
\begin{minipage}[t]{2.3in}
\vspace{0pt}
%\cellwidth{3em}
\gradetablestretch{2}
\vqword{Problem}
\addpoints % required here by exam.cls, even though questions haven't started yet.	
\gradetable[v]%[pages]  % Use [pages] to have grading table by page instead of question
\end{minipage}
\newpage % End of cover page

%%%%%%%%%%%%%%%%%%%%%%%%%%%%%%%%%%%%%%%%%%%%%%%%%%%%%%%%%%%%%%%%%%%%%%%%%%%%%%%%%%%%%
%
% See http://www-math.mit.edu/~psh/#ExamCls for full documentation, but the questions
% below give an idea of how to write questions [with parts] and have the points
% tracked automatically on the cover page.
%
%
%%%%%%%%%%%%%%%%%%%%%%%%%%%%%%%%%%%%%%%%%%%%%%%%%%%%%%%%%%%%%%%%%%%%%%%%%%%%%%%%%%%%%
\begin{questions}
	
% Question 1
\addpoints
\question[3] 세 다항식 $A=x^2+4x-2$, $B=2x^3+3x$, $C=-x^3+2x^2+4x-5$에 대하여 $A+B-C$의 값은 무엇인가? 

\begin{choices}
	\choice $4x^3-2x^2+3x+3$
	\choice $4x^3+2x^2+3x+3$
	\choice $4x^3-2x^2-3x+3$
	\choice $4x^3+2x^2-3x-3$
	\choice $4x^3-2x^2-3x-3$
\end{choices}
\vspace{1.0in}
% Question 2
\addpoints
\question[4] $(x-2))(x+2)(x^2+4)(x^4+16)$을 전개한 것은?
\begin{choices}
	\choice $x^4-16$
	\choice $x^6+64$
	\choice $x^6-64$
	\choice $x^8+256$
	\choice $x^8-256$
\end{choices}
\vspace{1.0in}

\newpage

\textbf{서술형 문항}
% 서술형 1번
\addpoints
\question[6] 사차방정식 $x^4-9=0$에 대하여 다음 각 물음에 답하시오.
\noaddpoints % If you remove this line, the grading table will show 20 points for this problem.
\begin{parts}
\part[2] 실근을 모두 구하시오.
\vspace{2in}
\part[2] 허근을 모두 구하시오.
\vspace{2in}
\part[3] 두 실근을, $\alpha$, $\beta$라 할 때, $\alpha+1$, $\beta+1$을 두근으로 갖는 이차방정식을 구하시오.
\vspace{2in}
\end{parts}

\newpage
% 서술형 2번
\addpoints
\question[8] 두 다항식 $f(x)$, $g(x)$에 대하여 다항식 $f(x) + g(x)$를 $x+2$로 나누었을 때의 나머지가 5이고, 다항식 $f(x)g(x)$를 $x+2$로 나누었을 때의 나머지가 3이다. 이때 다항식 ${f(x)}^3+{g(x)}^3$을 x+2로 나누었을때의 나머지를 구하고 그 과정을 서술하시오.
\vspace{4.0in}

% 서술형 3번
\addpoints
\question[10] 가로, 세로, 높이가 각각 $a$, $b$, $c$인 직육면체에서 모서리의 길이의 합을 $m$, 겉넓이를 $S$, 대각선의 길이를 $l$이라 할 때, 다음을 읽고 알맞은 답을 서술하시오.
\noaddpoints

\begin{parts}
	\part[2] m을 $a$, $b$, $c$에 대한 식으로 나타내시오.
	\vspace{2in}
	\part[2] $S$를 $a$, $b$, $c$에 대한 식으로 나타내시오.
	\vspace{2in}
	\part[2] $l$을 $a$, $b$, $c$에 대한 식으로 나타내시오.
	\vspace{2in}
	\part[4] 이차방정식 $3x^2-\frac{m}{2}x+\frac{S}{2}=0$이 중근을 가질 때, $\frac{m}{l}$의 값을 구하고, 그 과정을 서술하시오.
	\vspace{2in}
\end{parts}
	
%서술형 4번
\addpoints
\question[8] 상수 $k$와 집합 $A=\{a+bi$ $\vert$ $a^2+b^2=k$, a와 b는 실수$\}$에  대하여 $a+bi \in A$이면 $\frac{1}{a+bi} \in A$일 때, $k$의 값을 구하고 그 과정을 서술하시오.
\end{questions}
\end{document}