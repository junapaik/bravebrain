% These lines can probably stay unchanged, although you can remove the last
% two packages if you're not making pictures with tikz.
\documentclass[11pt]{exam}
\RequirePackage{amssymb, amsfonts, amsmath, latexsym, verbatim, xspace, setspace}
\RequirePackage{tikz, pgflibraryplotmarks}

% By default LaTeX uses large margins.  This doesn't work well on exams; problems
% end up in the "middle" of the page, reducing the amount of space for students
% to work on them.
\usepackage[margin=1in]{geometry}
\usepackage{kotex}


% Here's where you edit the Class, Exam, Date, etc.
\newcommand{\class}{SUPERNOVA 고1반}
\newcommand{\term}{2015 봄학기 중간고사}
\newcommand{\examdate}{4/10/2015}
\newcommand{\timelimit}{50 분}

% For an exam, single spacing is most appropriate
\singlespacing
% \onehalfspacing
% \doublespacing

% For an exam, we generally want to turn off paragraph indentation
\parindent 0ex

\begin{document} 
	% These commands set up the running header on the top of the exam pages
\pagestyle{head}
\firstpageheader{}{}{}
\runningheader{\class}{\examnum\ - Page \thepage\ of \numpages}{\examdate}
\runningheadrule
	
\begin{flushright}
	\begin{tabular}{p{2.8in} r l}
		\textbf{\class} & \textbf{Name (Print):} & \makebox[2in]{\hrulefill}\\
		\textbf{\term} &&\\
		\textbf{\examnum} &&\\
		\textbf{\examdate} &&\\
		\textbf{시험시간: \timelimit} 
	\end{tabular}\\
\end{flushright}
\rule[1ex]{\textwidth}{.1pt}

문제를 잘 읽고 답을 골라 답란에 표기하시오.\\
	
서술형 문항은 별도의 서술형 문항 답안지에, 검정색이나 파란색 필기구를 사용하여 작성하시오.\\
	
총문항수:  선택형 (16) 문항, 서술형 (4)문항\\
	
\begin{minipage}[t]{2.3in}
\vspace{0pt}
%\cellwidth{3em}
\gradetablestretch{2}
\vqword{Problem}
\addpoints % required here by exam.cls, even though questions haven't started yet.	
\gradetable[v]%[pages]  % Use [pages] to have grading table by page instead of question
\end{minipage}
\newpage % End of cover page

%%%%%%%%%%%%%%%%%%%%%%%%%%%%%%%%%%%%%%%%%%%%%%%%%%%%%%%%%%%%%%%%%%%%%%%%%%%%%%%%%%%%%
%
% See http://www-math.mit.edu/~psh/#ExamCls for full documentation, but the questions
% below give an idea of how to write questions [with parts] and have the points
% tracked automatically on the cover page.
%
%
%%%%%%%%%%%%%%%%%%%%%%%%%%%%%%%%%%%%%%%%%%%%%%%%%%%%%%%%%%%%%%%%%%%%%%%%%%%%%%%%%%%%%

\begin{questions}

% Question 1
\addpoints
\question[3] 세 다항식 $A=x^2+4x-2$, $B=2x^3+3x$, $C=-x^3+2x^2+4x-5$에 대하여 $A+B-C$의 값은 무엇인가? 

\begin{choices}
	\choice $4x^3-2x^2+3x+3$
	\choice $4x^3+2x^2+3x+3$
	\choice $4x^3-2x^2-3x+3$
	\choice $4x^3+2x^2-3x-3$
	\choice $4x^3-2x^2-3x-3$
\end{choices}
	
% Question 2
\addpoints
\question Consider the function $f(x)=x^2$.
\begin{parts}
\part[5] Find $f'(x)$ using the limit definition of derivative.
\vfill
\part[5] Find the line tangent to the graph of $y=f(x)$ at the point where $x=2$.
\vfill
\end{parts}

% If you want the total number of points for a question displayed at the top,
% as well as the number of points for each part, then you must turn off the point-counter
% or they will be double counted.
\addpoints
\question[10] Consider the function $f(x)=x^3$.
\noaddpoints % If you remove this line, the grading table will show 20 points for this problem.
\begin{parts}
\part[5] Find $f'(x)$ using the limit definition of derivative.
\vspace{4.5in}
\part[5] Find the line tangent to the graph of $y=f(x)$ at the point where $x=2$.
\end{parts}



\end{questions}
\end{document}